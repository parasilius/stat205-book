% !TEX root = ../pdf/stat205.tex
% [There are multiple stat205.tex files, but the one in ../pdf is the usual one]



%%%%%%%%%%%%%%%%%%%%%%%%%%%%%%%%%%%%%%%%%%%%%%%
\chapter{Probability and Accuracy~\label{ch:probandacc}}


\begin{verse}{\it
``When there are but two players, your theory which proceeds by combinations is very just. \\
But when there are three, I believe I have a proof that it is unjust that you should proceed in any other manner than the one I have.''\vspace*{6pt}} \\
\hspace*{2cm} -- Pascal's letter to Fermat\FOOTNOTE{from \url{https://www.york.ac.uk/depts/maths/histstat/pascal.pdf}}
\end{verse}
\vspace*{12pt}


\section{Indtroduction~\label{sec:intro}}

While scientists have always tried to understand the universe with technology and explain it with complete certainty, this is not always possible.
In other words, there is no other way but to accept chance as part of our lives.

The concept of chance has been extensively explored across elementary, professional, and philosophical literature, serving as a key motivation for our study.
While chance often appears unpredictable and devoid of structure, mathematicians have long sought to define it through rules and systematic frameworks.
Not surprisingly, gamblers were among the first to seek systematic frameworks for understanding their games - probing the mechanics of luck, wins and losses.
In 1654, a Parisian gambler named Antoine Gombaud (alias Chevalier de Méré), posed critical questions about winning probabilities to two of the era’s greatest mathematicians: Blaise Pascal and Pierre de Fermat.
Through their correspondence, the two mathematicians initiated the development of modern probability theory.
However, Gerolamo Cardano and Galileo Galilei, two Italian scholars, had also made significant contributions that captured the interest of Italian gamblers.

After years of development, the Russian mathematician Andrey Kolmogorov introduced the standard probability axioms in 1933, establishing the rigorous foundations of modern probability theory.
Today, probability theory serves as a fundamental tool across diverse fields including social sciences, medicine, biology, machine learning, physics, and countless other applications.
The following sections provide rigorous definitions of key concepts needed to establish a precise theoretical foundation.

\section{Random Experiment and Sample Space}

Suppose we want to conduct an experiment for which we know all possible outcomes.
Assuming the experimental conditions remain constant each time, if each realization of this experiment produces exactly one outcome, we call it a \keyterm{random experiment}.
We denote the set of all possible outcomes of such an experiment by \( S \), called the \keyterm{sample space}, which is a nonempty set.
\begin{exmp}\label{exmp:coin_toss}
	Consider the experiment of tossing a coin where the experimental conditions are controlled such that the coin lands on heads or tails, with no other possible outcomes.
	For instance the surface is chosen so that it won't land on edge.
	This is a random experiment in which we are interested in observing whether the coin lands heads or tails when viewed from above after it comes to rest
	Hence, denoting the outcome of observing heads by \( H \) and tails by \( T \), the sample space of this experiment is \( S = \{ H, T \} \).
\end{exmp}
\begin{exmp}\label{exmp:three_balls_urn}
	Consider an experiment where a ball is drawn from an urn containing one blue, one green, and one red ball.
	The experimenter cannot see inside the urn when making each draw.
	This is a random experiment in which we are interested in the color of the drawn ball.
	Hence, denoting the outcome of observing blue, green and red ball by \( B, G \) and \( R \) respectively, the sample space of this experiment is \( S = \{ B, G, R \} \).
\end{exmp}
\begin{exmp}
	Consider the experiment of drawing a card from a deck.
	Each time a card is drawn, it is returned to the deck and the duck is shuffled, so the conditions of each experiment remains the same.
	If we are interested in the suit of the drawn cards, the sample space is \( S = \{ \clubsuit, \vardiamondsuit, \spadesuit, \varheartsuit \} \).
	But if we are interested in the rank of the cards, the sample space becomes \( S = \{ A, 1, 2, 3, 4, 5, 6, 7, 8, 9, J, Q, K \} \).
\end{exmp}

In some cases, as the following example demonstrates, the sample space may be infinite:

\begin{exmp}\label{exmp:heads_observe}
	A coin is repeatedly tossed under the experimental conditions of Example \autoref{exmp:coin_toss} until heads appears.
	There are infinitely many possible outcomes:
	If the first coin toss results in heads, the experiment is immediately terminated.
	If it's tails, the coin is tossed again.
	The experiment terminates if heads appears for the second time.
	If not, the experiment continues until heads is observed.
	So the sample space in this case is \( S = \{ H, TH, TTH, TTTH, ... \} \).

	The first outcome is the case where heads appears on the first coin toss.
	The second outcome corresponds to heads appearing on the second coin toss, and so on.
\end{exmp}

\section{Event}

Each outcome of a random experiment is an element of its sample space, \( S \).
In doing these experiments, we are interested in observing some specific outcomes, or a subset of \( S \).
This subset is called an \keyterm{event} and is denoted by \( E \).
But we know that in each realization of a random experiment, only one element \( e \in S \) is observed.
If \( e \in S \), we say \( E \) has occured and if \( e \in S-E \), we say it has not occured.
\begin{exmp}\label{exmp:fair_die}
	In the experiment of throwing a fair die, \( S = \{ 1, 2, 3, 4, 5, 6 \} \).
	In case the experimenter is interested in the event of observing an even number, the event of interest would be \( E = \{ 2, 4, 6 \} \), a subset of \( S \).
	If instead they are interested in the event of observing an odd number, the event of interest would become \( E = \{ 1, 3, 5 \} \), again a subset of \( S \).
\end{exmp}
Note that in the previous example, we omitted certain details about the experimental conditions and our observations of interest.
Henceforth, unless explicitly stated otherwise, we make the following standard assumptions:
\begin{itemize}
	\item Each realization of a random experiment is performed under identical conditions.
	\item For a die roll, the random experiment of interest is observing the uppermost face after landing on the ground.
	\item For a coin toss, the random experiment of interest is whether it lands heads ( \( H \) ) or tails ( \( T \) ).
\end{itemize}
If \( E = S \), the event is called a \keyterm{sure event}, since each element of \( S \) is in \( E \) and so observing any outcome means the event has occured.
On the other hand, if \( E = \emptyset \), the event is called an \keyterm{impossible event}, since it contains no element at all.
\begin{exmp}
	In Example \autoref{exmp:fair_die}, observing a number greater than 6 is an impossible event, while observing a number less than 7 is a sure event.
\end{exmp}
An event consisting of only a single element of the sample space \( S \) is called an \keyterm{elementary event} or an \keyterm{atomic event}.
\begin{exmp}
	In Example \autoref{exmp:fair_die}, \( \{ 1 \}, \{ 2 \}, \{ 3 \}, \{ 4 \}, \{ 5 \} \) and \( \{ 6 \} \) are all elementary events.
\end{exmp}

\section{Set Theory Operations on Events}

We denoted the set of all outcomes of a random experiment by \( S \),
and we saw that we may be interested in a subset of this set, \( E \subset S \).
Now consider two events \( E, F \subset S \).

\subsection{Union of Two Events}

The \keyterm{union} of two events \( E \) and \( F \) consists of all elements that are in \( E \) or in \( F \) and is denoted by \( E \cup F \).
The occurrence of \( E \) or \( F \) results in the occurrence of \( E \cup F \).

\begin{exmp}\label{exmp:fair_die_union}
	In Example \autoref{exmp:fair_die}, we denote the event of observing an even number by \( E \)
	and the event of observing a prime number by \( F \).
	So \( E = \{ 2, 4, 6 \} \) and \( F = \{ 2, 3, 5 \} \).
	The union of these two events consists of all the elements in \( E \) or \( F \),
	meaning \( E \cup F = \{ 2, 3, 4, 5, 6 \} \).
	So if the outcome of a die roll is 4, we say the event of "observing a number which is even or prime" occurs,
	since 4 is an even number.
	
	If it's 5, we also say that this event occurs, since 5 is a prime number.
	
	Now suppose we observe 2 in a die roll. We again say that \( E \cup F \) occurs.
	The important thing to note here is that this \it{or} is an inclusive or, which means \( E \cup F \)
	is actually "observing a number that is even or prime or both".
	In this case, 2 is both an even and a prime number.
\end{exmp}

\subsection{Intersection of Two Events}

The \keyterm{intersection} of two events \( E \) and \( F \) consists of all elements in both \( E \) and \( F \) and is denoted by \( E \cap F \).
In order for \( E \cap F \) to occur, both \( E \) and \( F \) has to occur.

\begin{exmp}
	Consider the events \( E \) and \( F \) in Example \autoref{exmp:fair_die_union}.
	\( E \cap F \) is the event of "observing a number that is both even and prime".
	In other words, \( E \cap F \) consists of all elements that are in both \( E \) and \( F \),
	and so are both even and prime.
	Hence, \( E \cap F = \{ 2 \} \).

	So if 3 is observed in rolling a die, \( E \cap F \) does not occur since 3 is not even while it is a prime number.
	If 4 is observed, \( E \cap F \) again does not occur, since 4 is even but not prime.
	The only acceptable observation to say \( E \cap F \) occurs is 2, since only 2 is both even and prime among all the elements of \( E \) and \( F \).
\end{exmp}

If \( E \cap F = \emptyset \), we say the two events are \keyterm{incompatible}.
For instance, if we denote the event of "observing an odd number" by \( O \),
\( E \cap O = \emptyset \), meaning \( E \) and \( O \) are incompatible.
In other words, a single die roll cannot simultaneously result in both an even and an odd number.

\subsection{Difference of Two Events}

The \keyterm{difference} \( E - F \) is the event containing all elements in \( E \) but not in \( F \).

\begin{exmp}
	In Example \autoref{exmp:fair_die_union}, \( E - F \) is the event of "observing a number that is even but not prime".
	Thus \( E - F = \{ 4, 6 \} \).
	\( 3 \not\in E - F \) since it is prime, and \( 2 \not\in E - F \) because while it is an even number, it is also a prime number.
	So observing 3 or 2 upon rolling a die means \( E - F \) does not occur.
	For \( E - F \) to occur, we have to observe either 4 or 6.
\end{exmp}

\subsection{Complement of an Event}

The \keyterm{complement} of an event \( E \) is the difference \( S - E \), containing all alements in sample space that are not in \( E \).
The complement of E is denoted by \( E^\complement \), \( E' \) or \( \bar{E} \).

\begin{exmp}
	In Example \autoref{exmp:fair_die_union}, \( E^\complement \) is the event of "observing a number that is not even",
	which means \( E^\complement = \{ 1, 3, 5 \} \).
	So \( E^\complement \) occurs only when the observed number on the die is odd.

	Note that "observing a number that is not even" does not mean any number that is not even, but rather any number in the sample space \( S = \{ 1, 2, 3, 4, 5, 6 \} \) that is not even.
\end{exmp}

\section{What is Probability?}

There are three prominent ways to define probability.
In the three following subsections, these three definitions are explored.
Then an axiomatic definition of probability is given which is consistent with these three definitions.

\subsection{Classical Definition of Probability}

In this definition, each elementary event of the sample space is considered to be equally likely to occur.
For example in rolling a fair die, all the events \( \{ 1 \}, \{ 2 \}, \{ 3 \}, \{ 4 \}, \{ 5 \} \) and \( \{ 6 \} \) have the same probability.
Or in tossing a coin ( the random experiment in Example \autoref{exmp:coin_toss} ), the elementary events \( \{ H \} \) and \( \{ T \} \) are all equally likely to happen,
which is consistent with our intuition of 50\% chance of occurring for each event.

This interpretation of Probability goes back to Pierre-Simon Laplace, who wrote:
\begin{quote}
	The probability of an event is the ratio of the number of cases favorable to it, to the number of all cases possible when nothing leads us to expect that any one of these cases should occur more than any other, which renders them, for us, equally possible.
\FOOTNOTE{Laplace, Théorie analytique des probabilités, retrieved from \url{https://en.wikipedia.org/wiki/Classical_definition_of_probability}}
\end{quote}

While we can assign a probability to each elementary event when the sample space consists of finitely many elements,
it is not possible when the elementary events are not equally likely to occur or the sample space is infinite like we saw in Example \autoref{exmp:heads_observe}.

\subsection{Frequentist Definition of Probability}

Consider a randon experiment in which we are interested in the occurrence of event \( E \).
Suppose this experiment is repeated \( n \) times under identical experimental conditions,
and event \( E \) occurs \( r \) times in total.
\( r \) and \( \frac{r}{n} \) are said to be the \keyterm{frequency} and \keyterm{relative frequency} of \( E \) in these
\( n \) trials, respectively.

As the number of trials \( n \) increases, the frequency \( r \) and consequently the relative frequency \( \frac{r}{n} \) changes as well.
However, empirical observations show that the relative frequency converges to a constant value, which in this interpretation is defined as the probability of \( E \).



The frequentist interpretation of probability may trace its earliest conceptual origins to Aristotle, who wrote
\begin{quote}
	the probable is that which for the most part happens
	\FOOTNOTE{Aristotle, Rhetoric, retrieved from \url{https://en.wikipedia.org/wiki/Frequentist_probability}}
\end{quote}
The frequentist interpretation of probability rose to dominance during the 19th century, becoming the foundation of classic statistical inference.

Let us reconsider the random experiment described in Example \autoref{exmp:three_balls_urn}.
According to the classical interpretation of probability, the probability of drawing a blue ball is \( \frac{1}{3} \).
However, if we don't know the urn's contents, we can determine the probability of drawing a red ball through a simple experiment.
By repeatedly drawing balls from the urn and calculating the relative frequency of drawing blue balls,
we observe that this ratio converges to \( \frac{1}{3} \) as the number of trials increases,
which is consistent with the classical definition of probabiltiy.
But the frequentist interpretation offers another advantage:
Suppose there are two blue balls, one green ball and one red ball in the urn.
Here, while the sample space remains \( \{ B, G, R \} \), the elementary events now have different probabilities
because there is an additional blue ball in the urn!
So while we cannot calculate the probability of drawing a blue ball using the classical definition,
we can determine it through the frequentist approach by performing many trials and observing the convergence of relative frequency.

While this interpretation sounds like a better approach to defining probability,
it faces several limitations. 
Consider, for instance, estimating the probability of precipitation occurring tomorrow.
There is, in fact, only one tomorrow;
we cannot conduct multiple trials of "tomorrow" to count rainy occurrences and determine their relative frequency.
Moreover, maintaining truly identical experimental conditions is practically impossible in most real-world scenarios.
Another fundamental challenge lies in determining how many trials are sufficient for the relative frequency to converge to a stable probability value.