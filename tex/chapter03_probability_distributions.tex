% !TEX root = ../pdf/stat205.tex
% [There are multiple stat205.tex files, but the one in ../pdf is the usual one]



%%%%%%%%%%%%%%%%%%%%%%%%%%%%%%%%%%%%%%%%%%%%%%%
\chapter{Probability Distributions}


\begin{verse}{\it
``When there are but two players, your theory which proceeds by combinations is very just. \\
But when there are three, I believe I have a proof that it is unjust that you should proceed in any other manner than the one I have.''\vspace*{6pt}} \\
\hspace*{2cm} -- Pascal's letter to Fermat\FOOTNOTE{from \url{https://www.york.ac.uk/depts/maths/histstat/pascal.pdf}}
\end{verse}
\vspace*{12pt}


\section{Random Variables}

In Example \autoref{exmp:three_fair_coins}, we denote the number of observed heads by \( \bm{X} \),
which can take on values 0, 1, 2, or 3.
Each of these numbers corresponds to one of the following events:
\begin{itemize}
    \item 0: \( A = \{ TTT \} \),
    \item 1: \( B = \{ HTT, THT, TTH \} \),
    \item 2: \( C = \{ HHT, HTH, THH \} \),
    \item 3: \( D = \{ HHH \} \)
\end{itemize}
In other words, the sample space is partitioned into events \( A, B, C, \) and \( D \).
Hence, we can map each element of \( S \) to exactly one element of \( \{ 0, 1, 2, 3 \} \).
This mapping is achieved using \( \bm{X} \), which is called a \keyterm{random variable}.

Thus:
\begin{gather*}
    X(TTT) = 0, X(HTT) = X(THT) = X(TTH) = 1, X(HHT) = X(HTH) = X(THH) = 2, X(HHH)= 3
\end{gather*}

More formally, in a probability model with sample space \( S \),
a random variable (or simply a variable) is a real-valued function \( X: S \rightarrow R \),
where the range of \( \bm{X} \) is a subset of the real numbers.
The range of \( \bm{X} \) is called the \keyterm{support} of \( \bm{X} \) and is denoted by \( S_{\bm{X}} \).

Conventionally, if \( A \subset R\), we define:
\begin{gather*}
    (\bm{X} \in A) = \{ e \in S | \bm{X}(e) \in A \}
\end{gather*}
For instance, in the previous example, \( (\bm{X} < 2) = (\bm{X} \in (-\infty, 2)) = (\bm{X} \in \{ 0, 1 \}) = \{ TTT, HTT, THT, TTH \} \).

When working with random variables, probabilities can be described quantitatively.
For example, if we want the probability that the number of heads is fewer than 2, we are interested in the event \( (\bm{X} < 2) \).
As shown earlier, this corresponds to:
\begin{gather*}
    P(\bm{X} < 2) = P(\bm{X} \in \{ TTT, HTT, THT, TTH \}) = \frac{4}{8} = \frac{1}{4}
\end{gather*}

\begin{exmp}
    Suppose in Example \autoref{exmp:heads_observe}, \( \bm{X} \) is the number of coin tosses required to observe the first heads.
    Thus, the support of \( \bm{X} \) is \( S_{\bm{X}} = \{ 1, 2, \ldots \} \).
    For instance, if one iteration of this experiment yields the outcome \( TTTH \),
    then \( \bm{X}(TTTH) = 4 \), as the first heads occurs on the fourth toss.

    The probability that the number of coin tosses required to observe the first heads is an odd number is given by:
    \begin{align*}
        P(\bm{X} \in \{ 1, 3, \ldots \}) &= P(\{ H, TTH, \ldots \})\\
        &= P(\{ H \}) + P(\{ TTH \}) + \ldots\\
        &= (\frac{1}{2}) + (\frac{1}{2})^3 + \ldots\\
        &= \frac{2}{3}
    \end{align*}
\end{exmp}

\begin{exmp}
    In the previous example, what is the probabiltiy that more than six coin tosses are required to observe the first heads?
\end{exmp}

In the two examples we analyzed so far, the support \( S_{\bm{X}} \) is countable.
In this case, we say \( \bm{X} \) is a \keyterm{discrete random variable}.
Conversely, if a random variable can take on infinitely many (uncountable) number of values,
we call it a \keyterm{continuous random variable}.

Below are examples of continuous random variables:

\begin{exmp}
    In Example \autoref{exmp:lightbulb_lifespan}, let \( \bm{X} \) be lightbulb's lifespan.
    Since lifespans can take any non-negative real value, \( \bm{X} \) is a continuous random variable.
\end{exmp}

\begin{exmp}
    Suppose a needle is dropped at random into a circular disk of radius 3.
    The sample space consists of all possible landing positions of the needle, which are uncountable.
    Let \( \bm{X} \) denote the distance from the needle's landing point to the center of the disk.
    Then \( \bm{X} \) is a continuous random variable with support \( S_{\bm{X}} = [0, 3] \),
    where the interval includes the boundary points (accounting for the possibility of landing exactly on the disk's edge).
\end{exmp}